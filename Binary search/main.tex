\documentclass{beamer}
\usepackage{svg}
\usepackage{hyperref}
\usepackage{tikz}
\usepackage{mathtools}
\usepackage{wrapfig}
\usepackage{amsmath}
\usepackage{adjustbox}
\usepackage{graphicx}
\usepackage[T2A]{fontenc}
\usepackage[utf8]{inputenc}
\usepackage{booktabs,tabularx}
\usepackage{makecell}
\usepackage{minted}
\usepackage[utf8]{inputenc}
\usepackage[english,russian]{babel}
 
\usepackage{minted}
    \hypersetup{colorlinks=true,linkcolor = blue,
            allbordercolors={0 0 0},
            pdfborderstyle={/S/U/W 1}}

\renewcommand{\theFancyVerbLine}{
  \sffamily\textcolor[rgb]{0.5,0.5,0.5}{\scriptsize\arabic{FancyVerbLine}}}
\renewcommand\theadalign{bc}
\renewcommand\theadfont{\bfseries}
\renewcommand\theadgape{\Gape[4pt]}
\renewcommand\cellgape{\Gape[4pt]}
\newcommand{\mybluehref}[2]{\hyperref[#1]{\color{green}\setulcolor{red}\ul{#2}}}

\title{Бинарный поиск. Тернарный поиск}


\author{Вадим Бездушный}

\date{Подготовка к обласной олимпиаде, 2018}

% Let's get started
\begin{document}

\begin{frame}
  \titlepage
\end{frame}



\begin{frame}[fragile]{Бинарный поиск: ваш код}
    \begin{minted}{cpp}
    int L =             ;
    int R =             ;
    while (         ) {
        int M =            ;
        if (check(M)) {
            R =     ;
        } else {
            L =     ;
        }
    }
    \end{minted}
\end{frame}

\begin{frame}[fragile]{Бинарный поиск: мой код}
    \begin{minted}{cpp}
    int L = minValue - 1;
    int R = maxValue + 1;
    while (R - L > 1) {
        int M = (L + R) / 2;
        if (check(M)) {
            R = M   ;
        } else {
            L = M   ;
        }
    }
    \end{minted}
\end{frame}

\begin{frame}[fragile]{Бинарный поиск: код для массива}
    \begin{minted}{cpp}
    int L = -1;
    int R = a.length();
    while (R - L > 1) {
        int M = (L + R) / 2;
        if (a[M] >= key) {
            R = M   ;
        } else {
            L = M   ;
        }
    }
    \end{minted}
\end{frame}


\begin{frame}[fragile]{Вещественный Бинарный поиск: ваш код}
    \begin{minted}{cpp}
    double L = minValue;
    double R = maxValue;
    
        double M = (L + R) / 2;
        if (check(M)) {
            R = M   ;
        } else {
            L = M   ;
        }
    }
    \end{minted}
\end{frame}

\begin{frame}[fragile]{Вещественный Бинарный поиск: мой код}
    \begin{minted}{cpp}
    double L = minValue;
    double R = maxValue;
    for(int iter = 0; iter < 100; iter++){
        double M = (L + R) / 2;
        if (check(M)) {
            R = M   ;
        } else {
            L = M   ;
        }
    }
    \end{minted}
\end{frame}


\begin{frame}[fragile]{Поиск границ}
    \begin{minted}{cpp}
    double leftBorder = -1
    while(f(leftBorder))
        leftBorder *=2;

    double rightBorder = 1
    while(!f(rightBorder))
        rightBorder *= 2;
    \end{minted}
\end{frame}



\begin{frame}[fragile]{Тернарный поиск}
\begin{minted}{cpp}
double ternarySearchMin(double left, double right){
    while(right - left > eps){
    
        double midL = (left * 2 + right) / 3;
        double midR = (left + right * 2) / 3;
        if f(midL) < f(midR)
            right = midR;
        else
            left = midL;
    }
    return (left + right) / 2;
}    
\end{minted}
\end{frame}

\begin{frame}[fragile]{Тернарный поиск: мой код}
\begin{minted}{cpp}
double ternarySearchMin(double left, double right){
    for(int iter = 0; iter < 300; iter++){
        double range = (right - left) / 3;
        double midL = left  + range;
        double midR = right - range;
        if f(midL) < f(midR)
            right = midR;
        else
            left = midL;
    }
    return (left + right) / 2;
}    
\end{minted}
\end{frame}


\begin{frame}{Задача}
\url{https://www.e-olymp.com/uk/problems/3968}

Знайдіть таке число $x$, що $x^2 + \sqrt{x} = C$ з точністю не менше 6 знаків після крапки
\end{frame}


\begin{frame}{Приближенный двоичный поиск}
\url{http://informatics.mccme.ru/mod/statements/view3.php?id=192&chapterid=2}
\end{frame}


\begin{frame}{Дипломы}
\url{http://informatics.mccme.ru/mod/statements/view3.php?id=1966&chapterid=1923}
\end{frame}

\begin{frame}{Очень Легкая Задача}
\url{http://informatics.mccme.ru/mod/statements/view3.php?id=1966&chapterid=490}
\end{frame}


\begin{frame}{Мотузки}
\url{https://www.e-olymp.com/uk/problems/3967}
\end{frame}

\begin{frame}{Субботник}
\url{http://informatics.mccme.ru/mod/statements/view3.php?id=1966&chapterid=1620}
\end{frame}


\begin{frame}{Поляна дров}
\url{https://www.e-olymp.com/uk/problems/5182}
\end{frame}



\end{document}